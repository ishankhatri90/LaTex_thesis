%%%%%%%%%%%%%%%%%%%%%%%%%%%%%%%%%%%%%%%%%%%%%%%%%%%%
%
%  New template code for TAMU Theses and Dissertations starting Fall 2012.  
%  For more info about this template or the 
%  TAMU LaTeX User's Group, see http://www.howdy.me/.
%
%  Author: Wendy Lynn Turner 
%	 Version 1.0 
%  Last updated 8/5/2012
%
%%%%%%%%%%%%%%%%%%%%%%%%%%%%%%%%%%%%%%%%%%%%%%%%%%%
%%%%%%%%%%%%%%%%%%%%%%%%%%%%%%%%%%%%%%%%%%%%%%%%%%%%%%%%%%%%%%%%%%%%%
%%                           ABSTRACT 
%%%%%%%%%%%%%%%%%%%%%%%%%%%%%%%%%%%%%%%%%%%%%%%%%%%%%%%%%%%%%%%%%%%%%


\chapter*{ABSTRACT}
\addcontentsline{toc}{chapter}{ABSTRACT} % Needs to be set to part, so the TOC doesn't add 'CHAPTER ' prefix in the TOC.

\thispagestyle{plain} % No headers, just page numbers
\pagenumbering{roman} % Roman numerals
\setcounter{page}{3}

\begin{center}

\pvamumanuscripttitle .

(\pvamugradmonth \hspace{2pt} \pvamugradyear)

\vspace{15pt}

\pvamufullname, B.Eng., Lalbhai Dalpatbhai College of Engineering

Chair of Advisory Committee: Dr. Lijun Qian

\par\end{center}

\justify

\indent 
%\justify 

	Demand forecasting has become very significant in recent years because of its accurate and efficient resource allocation for companies. However, demand forecasting usually involves processing very large and sophisticated data sets and it is very challenging to achieve accurate prediction. Although there are many existing methods in the literature, they could not be directly applied to big data. In this thesis, a novel demand forecasting method using deep learning specifically, a Convolutional Neural Network (CNN) based model is proposed to process huge amounts of data. A unique feature of the proposed method is that it has adaptable computational intricacy even with the increasing dimensionality of the data. As a result, the proposed method has excellent scalability. Large data sets, namely the usage data of Bike Sharing Service in New York City are used to validate the proposed method. The goal is to predict hourly bike rental demand in every station of New York City to improve the service and environment, and eventually build a ``Green" city. The results demonstrate the effectiveness and superior performance of the proposed scheme. This is also one of winning strategies of the student team from the CREDIT Center in the 2016 IEEE Big Data Analytics Competition organized by the IEEE Big Data Initiative.

%In modern age and time, the term "Big Data" is one of the most discussed topic. It is also currently a massive challenge with many applications and has become the digital resource to power our real world and is anticipated to continue to do so in foreseeable future. In order to deal with this challenge, the method Machine Learning is becoming more and more popular. This method indicates to automated detection in the data to find meaningful patterns. In this thesis we discuss how can we use Machine learning for doing demand forecasting and propose a model for transportation demand forecasting and results also showes the efficiency of the method. This forecasting is deeply relying on understanding the data and identifying the problem, prospective, confronts and specially related applications.
%
%In order to disseminate knowledge IEEE Big Data Initiative along with CyberC prepared a competition of Big Data analytics where the goal was to increase the usage of Bike Sharing Service in New York City, and the idea of this thesis came from winning the competition which provided the dataset of City Bike share data of New York City. In order to solve the problem I have used historic bike usage with the weather data from another domain and treat those as a time series problem and attempted to compare it with a image processing problem by using Convolutional Neural Network (CNN) in order to forecast hourly bike rental demand in NYC. This thesis absorbed a lot of knowledge from the competition and try to find a model with the lowest error rate. We also then compared our model by building three other very popular baseline methods in order to show the differences in prediction accuracy. This thesis in overall asks a simple question ``How many bikes in every hour can meet users requirement?". In future I would like to treat every station individually and forecast their bike demands separately. Therefore, the truthful meaning of this thesis is to improve the bike share rebalance problem by forecasting the bike demand in hourly basis and to find a way to give the city bike users a better experience overall. 



 

\pagebreak{}







