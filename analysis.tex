%%%%%%%%%%%%%%%%%%%%%%%%%%%%%%%%%%%%%%%%%%%%%%%%%%%
%
%  New template code for TAMU Theses and Dissertations starting Fall 2012.  
%  For more info about this template or the 
%  TAMU LaTeX User's Group, see http://www.howdy.me/.
%
%  Author: Wendy Lynn Turner 
%	 Version 1.0 
%  Last updated 8/5/2012
%
%%%%%%%%%%%%%%%%%%%%%%%%%%%%%%%%%%%%%%%%%%%%%%%%%%%

%%%%%%%%%%%%%%%%%%%%%%%%%%%%%%%%%%%%%%%%%%%%%%%%%%%%%%%%%%%%%%%%%%%%%%%
%%%                           SECTION II
%%%%%%%%%%%%%%%%%%%%%%%%%%%%%%%%%%%%%%%%%%%%%%%%%%%%%%%%%%%%%%%%%%%%%%

\chapter{\uppercase { Conclusion \& Future work}}

\label{Conclusion}

This chapter will review the whole thesis and summarize the current need of demand forecasting and the approach of bike sharing service demand forecasting. Few future work and research will also be mentioned. 


\section {Conclusion}
\label{conclusion}

In the years to come ``Big Data" term will continue to grow bigger and bigger, which makes it essential for companies to embrace this challenge with open arms and learning, implementing the required skills to tackle this. This thesis, which is mainly focuses on the demand forecasting in different fields, first recognized few challenges it poses and showing potential of using it for the benefit of organizations. It also a must to dedicate enough time and resources to be able to overcome the issues. 




This thesis is specially looking at a particular transportation demand forecasting by doing hourly bike rental demand forecasting where the idea of the topic came from winning the Big data competition where the main goal was to improve the bike share program (as mentioned earlier). In order to do the bike rental demand forecasting first we did some statistical analysis on the bike rental data. From those analyses it is found that there is a big prospective of trying to attract more female users in to this bike share program. Also the visualization of the different factors of the data also showed customers mainly uses bike share program for doing short trips which makes it very important to have rentable bikes available or free parking spaces in the stations whenever customer wants. For the first problem, different types of bikes according to gender preference and forecasting hourly rental bike demand using CNN is for the latter problem is proposed. The main contribution of this thesis is from the results gained from the experiment, which proves that doing forecasting using CNN and treating bike samples as image samples to solve regression problem is a very reasonable and achievable way to deal with the problems stated above. Nevertheless, CNN performs better than the baseline models but it is not significant. Reason being the lack of datasets may reduce the performance of the overall model. 

\section{Future Work}
\label{future}

There are many additional work can be done in this topic as demand forecasting is proving to be one of the most important area for many organizations. 

In future a demand forecasting experiment on the health care program can be carried out as it is also a very important section of demand forecasting. 

For the transportation demand forecasting in bike share program, there are too many factors which may cause variance in user's behavior such as - biking events or marathon, extremely snowy day or tornado, government restriction regarding traffic. Therefore, it is almost impossible to consider all the factors in one research, thus, this thesis focuses on building a model to forecast the hourly rental bike demand to predict the number of rental demand in future based on the historical data of bike rentals, weather and holiday data. So separating every docking stations and treating them individually and forecasting their rental demands can be another very useful way to solve this problem. 

The results also showed inconsistency between the different layers of CNN so finding the best structure needed for the different datasets is also a very important task to work on. It is also possible to implement transfer learning on the bike rental data forecasting model from another transportation forecasting model such as taxi or uber data. Maybe doing this we can get more insights on the users behavior which will make our model more useful. 

Finally, ``Big Data", as we all know the bigger the dataset is better the performance will be. So collecting more data can only improve the forecasting results and predicting accuracy. But it is also important to clear out any noises it may have in a very tactful way to get the best possible results of demand forecasting. 











