%%%%%%%%%%%%%%%%%%%%%%%%%%%%%%%%%%%%%%%%%%%%%%%%%%%
%
%  New template code for TAMU Theses and Dissertations starting Fall 2012.  
%  For more info about this template or the 
%  TAMU LaTeX User's Group, see http://www.howdy.me/.
%
%  Author: Wendy Lynn Turner 
%	 Version 1.0 
%  Last updated 8/5/2012
%
%%%%%%%%%%%%%%%%%%%%%%%%%%%%%%%%%%%%%%%%%%%%%%%%%%%
%%%%%%%%%%%%%%%%%%%%%%%%%%%%%%%%%%%%%%%%%%%%%%%%%%%%%%%%%%%%%%%%%%%%%%
%%                           SECTION III
%%%%%%%%%%%%%%%%%%%%%%%%%%%%%%%%%%%%%%%%%%%%%%%%%%%%%%%%%%%%%%%%%%%%%
%\setlength{\parindent}{2em}
%\setlength{\parskip}{1em}

\chapter{\uppercase{Literature Review}}

\section{ Theory}
\label{DST}
As previously mentioned, forecasting using Big Data is a ground-breaking sensation which is  one of the most discussed and one of the most important topic in the modern age. In this section a detailed review of the use of Big Data for forecasting with identifying the problem, potential \& challenges come with it also the related applications is presented. For obtaining meaningful forecasting results from Big Data is often faced with a lot of challenges such as- skills used, software and hardware, architecture of the algorithm, signal to noise ratio, statistical importance as well as the nature of the data itself. Even though all this challenges are continuously trying to impede the forecasting, many fields like Energy, Economics, Population dynamics are the main exploiters of it. With some common tools like Bayesian, Regression, Neural Network models are implemented for these type of forecasting with the availability of Big Data. 


\subsection{The Basics}
\label{Basics}

The arrival of Big Data is becoming history now as industries around the world are generating enormous amounts of data every second. Therefore it is very important for organizations to respond to this digital growth of asset and adopt necessary tools to exploit it for their own benefit. As~\cite{varian} states that there is a need for adoption of data mining techniques which can support in modelling the complicated connections of the characteristics present in Big Data. Also in current years, organization felt the importance of the risk management due to the recent financial crisis. According to~\cite{silva} corporations are now trying to minimize associated threats of maximizing their opportunity by using risk management tools. Hence, forecasting using Big Data has the potential and the ability to boost the organizational performance whereas empowering enhanced risk management~\cite{brown2011you}.
Conversely, Not every author is agreeing with the Big Data sensation. In~\cite{KAUFMAN199629} Poynter describes that Big Data will be more intuitive for connecting dots rather than painting a whole new picture. According to~\cite{walker} the year 2013 was the year to get familiarized with Big Data and 2014 is the year to start exploiting the Big Data for profitable gains. This chapter will provide a brief knowledge about Demand Forecasting in current time using real world Big Data also the challenges and applications of it. 


\section{``Potential"  and  ``Problem"}
\label{PaP}


Demand forecasting using Big data has a lot of different opportunities for lucrative gain. Companies have already started exploiting Big Data in order to forecast demand and work accordingly in order to improve better management and also increase revenues. Fashion companies like EDITD are trying to forecast future fashion trend demands by using the data collected from social media. Netflix used Big Data for decision making before starting their own TV show ‘House of Cards’ which then became very successful. Airline companies are also relies on the demand forecasting very much. Therefore, we can clearly see that the potential behind demand forecasting using Big Data are huge and actually overwhelming. But this is so astonishing that sometimes it becomes scary as in~\cite{duhigg2012companies} author described a story where a woman complained about a ‘Target’ store for sending her high school daughter pregnancy related coupons. But a few weeks later she apologized as her daughter was in fact pregnant~\cite{duhigg2012companies}. 


\section{Challenges}
\label{Challenges}

Demand forecasting using Big Data poses a lot of challenges which we must overcome in order get any meaningful outcomes from the data. It is also vital to know that Big Data's obtainability does not end the problems~\cite{ba}, therefore, in this section I will be primarily focusing on the challenges. Some of the challenges are associated with the techniques, assumptions~\cite{silver2012signal} while~\cite {west_2013} suggests that absence of theoretical support of Big Data is also a major concern. 


\subsection{Skills}
\label{Skills}

One of the major challenges while trying to forecast demand using Big Data is the expertise needed and the availability of experts in this specific tasks.~\cite {arribas2014accidental} shows that the importance of cutting-edge skills in order to deal with Big Data and~\cite{wesson2014big} states the shortage of the data scientists who are able to tackle this problem. Traditional researchers and scientists are working on old methods for decades and now this emerging era of Big Data is a challenge in itself. Therefore, inappropriateness of these traditional methods which were used on the old data types are hampering current world from forecasting using Big Data~\cite{skupin20081}~\cite{arribas2014accidental}. For more than 50 years these traditional statisticians are working on those outdated techniques, which makes it very difficult to develop new skills for Big Data forecasting~\cite{einav2014data}. Hence, it is necessary for current big educational organizations all over the world to modify their curriculum to adopt the need for new generations of statisticians by include the skills needed for understanding, analyzing and forecasting with Big Data. Thus we can make sure that in future we will have better skills to tackle this new very powerful asset of Big Data and also will be able to forecast more important insights from the data. 



\subsection{Unwanted Data / Noise}
\label{noise}

Unwanted data or noise is actually making it very difficult to forecast something accurately using Big Data. In~\cite{silver2012signal} the author points out at this problem saying an increasing noise to signal ration is present in the current Big Data. As the data gets larger and larger, exploiting this data is also getting more difficult. Before the era of Big Data, traditional methods used to work relatively well in those smaller sets of data set, but now the Big Data sets consists of a lot of unwanted data or wrong data which makes forecasting very complex. Therefore, New techniques and methods needs to be applied in order to filter out those noises and restructure the data in a way so that we can get the most out of it. There are few techniques already present like- singular spectrum analysis which tries to filter out the noises from a time series and reform the time series with much less noise, which is also proved in ~\cite{hassani2009forecasting} ~\cite{hassani2015forecasting}.

\subsection{Hardware and Software}
\label{Hardsoft}

Almost all of us who is involved with statistical analysis have faced the problem that programs crashes after doing few thousands to iteration. Like~\cite{arribas2014accidental} says existing software to do statistical analysis are not powerful enough to do Big Data forecasting where~\cite{needham2013disruptive} says supercomputers are needed. Capable software’s are also as important as hardware’s and nowadays a lot of forecasting techniques are developed and some of them can automatically forecast problems~\cite{hyndman2013forecasting}. But these programs are not as reliable when it comes to deal with Big Data. Therefore, to make a powerful forecasting software it needs the required computing capability and only then we can effectively deal with forecasting as the data is getting bigger and bigger. 

\subsection{Algorithm Architecture}
\label{Algo}

Current data mining methods are often used for smaller datasets and are unable to perform in a satisfactory manner. Also many time this technique can not deal with the data which are not present in their main memory so data movement between locations are often poses a big problem~\cite{jadhav2013big}. Parallel programming or lambda architecture mentioned in~\cite{marz2015big} are samples of current ongoing research happening on this topic.

\subsection{Statistical Importance}
\label{Statistics}

While doing demand forecasting or forecasting with Big Data, it is very easy to make incorrect findings. The amount and the randomness of the data and the noise available in it increases the chance of getting a misleading forecasting report. On top of that finding an appropriate technique has always been a very difficult job for big data. In~\cite{lohr2012age} author also talks about the threat of making false discoveries. 

Apart from all these the phrase Big Data itself is a very difficult challenge while doing demand forecasting, due to its in-built characteristics. Very complex and unstable structure of the data makes it tough for data scientists to get any forecasting which is free from poor out of sample outcome~\cite{einav2014data}. Also it is very difficult to chose the correct method and while using deep learning it is very difficult to chose the number of hidden layers it should have due to the shapelessness of the data as it is continuously changing in real time. 



\section{Applications of Demand Forecasting}
\label{Application}

In this segment, we introduce the present applications which uses demand forecasting by exploiting data mining techniques on Big Data. The main reasons behind doing demand forecasting are :


\begin{itemize}
\item Cost reduction
\item Agility in the market
\item Estimating the financial capability
\item Planning ahead to meet necessities
\item Improve accuracy
\item Visualization - Relationships between different factors


\end{itemize}


\subsection{Energy Demand Forecasting}
\label{Energy}

Smart meters and advanced technology in the field of energy are the reason behind the huge data sets that are available currently. These Big Data have opened the door towards endless opportunity for improvement in better energy management and conservation system as well as new energy systems. A lot of experiments have been carried out in the field of energy like, in~\cite{nguyen2010short} Nguyen and Nabney tried to exploit the Big Data from the British Energy organizations and they used few different models to evaluate the wavelet transform like, linear regression, GARCH, MLP and Radial basis functions. They found that MLP or GARCH with wavelet transform are the best models for forecasting electricity demands as well as the gas price. 


\subsection{Health Care Demand Forecasting}
\label{Health}

By using the Big Data in health care field it is possible to improve charge capture and also can reduce the denial rate of reimbursement. According to~\cite{callahan2004effective} it is also possible to do location based forecasting in order to coordinate surgeon preference information and also forecasting demand based on the patient demographics, clinic scheduling as well as seasonal demands. These results provide lower cost for the preparations of cases, reduces inventory levels in OR, supports clinics for much better planning and budgeting to introduce new or to expand current surgical agendas also improving the service and fill rates which leads those clinics to improve patient satisfaction, improved rating and productivity as well as patient safety. 

%\subsection{Tourism Demand Forecasting}
%\label{Tourism}

\subsection{Transportation Demand Forecasting}
\label{Transportation}
Demand forecasting in transportation is the method to predict the the number of people or the transportation mediums will be used in the future. This forecasting methods can then be applied on a range of different situations from forecasting traffic volumes on a particular road on a given day to forecasting passenger counts on trains or to forecasting number of ships will be there in a port. The reason behind these forecast is to predict what will be the demand in the future might be and also to specify the standard for planning and design to the most efficient transportation system. Application of demand forecasting in transportation allows us to improve overall transportation system, planning ahead to meet the requirements, developing the infrastructure capacity and design estimation. These can then make sure the approximation of the financial capability of the system as well as the environmental effects it has on them.   




In this thesis, I have used the Bike sharing system data of the New York city in order to forecast the demand of bike sharing service and to find ways to improve the service so that clients can get better experience from it. 

\section {Transportation Demand Forecasting : Bike Share Service Approach}

Forecasting ride sharing system usage is becoming very popular with the growing of industries like Uber-lyft. But substantial traffic in busy cities and aspiration of an environmental friendly way of transportation made bike share system very popular and attractive in recent years. Thus Bike share systems provide users an alternate and ecological way of conveyance by dodging personal car usage or taxis and staying stuck at traffic for a long period of time. This growing demand of bike sharing system making big cities to open their own bike share service for which currently more than 600 cities all over the world has the bike share program, allowing their residents to go from one place to another by renting bikes. Jensen et al.~\cite{jensen2010characterizing} also showed (in his case for Lyon) that sometimes it is faster than using personal car to go to places. Except the expenditure almost all the bike sharing program shares the similar technologies and it will be really exciting to predict the future usage of this amazing program and thus allowing users to plan their everyday trip more accurately with satisfaction. 

In May 2013, Citibank started a bike sharing service in New York City which is called City Bike. This is the nations leading bike share service with more than 10,000 bikes with 600 stations across the city where customers can rent a bike from any of those station for a quick, fun and reasonable ride on an as-needed basis. Users can obtain membership to rent the bikes and return the bike in an automated process via a network of stations throughout the city. City Bike share recently hit 10 million trips in a year and thus the system generates a lot of data which is very eye-catching field for researchers as it explicitly contains data about the departure station locations, age group and gender of the users, duration, arrival location and so on. Therefore, this service works like sensor network and can be used to study the movement flexibility in a particular city.  


Researchers around the world did some related research on the bike sharing system where they all tried to predict that a user can find a bike whenever he goes to the station. 



\subsection{Related Works}
\label {related}

Bike sharing system first started in Amsterdam in 1965 with the 1st generation `White bikes' and since then the popularity is increasing all over the world~\cite{demaio2009bike}.  Due to technological advancement the 3rd generation bike sharing system is available now, where it is possible to do bicycle reservations, information tracking, automated pickup and dropoff system~\cite{shaheen2010bikesharing}. This IT based integrated system for bike share programs made it possible to collect its data to do research and find possible ways to improve the system. 

Researchers around the world did some related research on the bike sharing system and these are mainly focused into four different problems. Strategic program scheme, quality improvement, demand analysis and rebalancing issue.
Almost all the major cities around the world already planned or is planning or considering execution of this bike share system. For the first problem, estimating potential demand in order to optimize location, deel’ollio et al.~\cite{moura2011implementing} presented an inclusive approach for implementation. Location is also another issue which was widely discussed in Martinez et al.~\cite{martinez2012optimisation} and Prem Kumar et al.~\cite{kumar2012optimizing} where they propose MIP models to tackle this issue. Also a trade off model for the interest of both customers and sponsors were proposed in~\cite{lin2011strategic}.

There has been also several research focusing on the second problem which is the quality improvement. Raviv et al.~\cite{raviv2013static} proposes a model for probable user’s displeasure in every station based on a queuing system with finite capacity. Nair et al.~\cite{nair2013large} and Nair and Miller~\cite{miller2010fleet} decompose system-wide reliability into a set of dual-bounded chance constraints for every station. By assumption of waiting customers~\cite{leurent2012modelling} shows bike sharing stations as a dual Markovian waiting system. 
 
Similar to this thesis there has been few research about Demand analysis where the analysis is mainly divided into two sections – 1) forecasting future demand, 2) understanding factors for administrative decisions. Kaltenbrunner~\cite{kaltenbrunner2010urban} tries to predict the system inventory state also suggests making this information available to users. Rixey et al.~\cite{rixey2013station} investigates the effects of demographic and built environment characteristics near three bike sharing stations on bike sharing ridership levels in three operational U.S. systems. The regression analysis identifies a number of variables as having statistically significant correlations with station-level bike sharing ridership and identify station locations that will serve the greatest number of riders. Borgnat and Abry~\cite{borgnat2011shared} , Froehlich and Oliver~\cite{froehlich2008measuring}, Lathia et al.~\cite{lathia2012measuring} and Borgnat et al.~\cite{borgnat2009studying} identify a temporal demand pattern and forecast the number of rentals. Giot and cherrier~\cite{giot2014predicting} attempts to predict the bike share usage up to one day ahead with an hour of granularity and compared how adaboost regressor, ridge regression, SVR, random forest nad gradient boosting regressor performs on dataset on two years. Their results sows that most regressors are sensitive to over-fitting. 

For the fourth problem, Shu et al.~\cite{shu2010bicycle} showed that estimate rebalancing operations can lead to an additional 15-20\% of trips supported system-wide. Chemla et al.~\cite{chemla2013bike} present a branch-and-cut algorithm for the single-vehicle problem, with results on instances of up to 100 stations. Lin and Chou~\cite{lin2012geo}, rather than simply using Euclidean distance they proposed an optimization way taking road conditions, traffic regulations, and geographical factors into account. Their actual distance path calculation is used to implement a exploratory for the VRP. Naturally, using actual distances would lead to decreased costs in practice. Yoon et al.~\cite{yoon2012cityride} proposes auto regressive moving average model to help bike share users to navigate around the city. By giving the starting and end point the application can suggest which stations would be best to use by considering the walking time and biking time and available return slots at the end station. This model is useful for stationary signals and the bike share system which is varying over time.  

From the researches mentioned above it can be observed that there are many research going on in different topics related to bike share system but for the forecasting bike usage people are mainly using machine learning models with space and time series features and building applications to help customers navigate around the city easily. 
For this thesis, I present a Convolutional Neural Network based demand forecasting model where the bike usage data is treated as images and solved a regression problem using CNN. Results also shows the efficiency of the model. 



